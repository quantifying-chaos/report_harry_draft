\begin{abstract}
Chaos is an universal phenomenon observed in every aspect of the physical world. Examples include the weather system, the celestial movements, demographic evolutions, and economic activities. 
The descriptions thereof often comprise of the adjective ``unpredictable,'' ``disordered'', ``entropic,'' ``confusing'', subject on to chances, and, in summary, ``chaotic''.
It may be of surprise that much more can be said to the even the simplest example.
In this report, however, we will demonstrate there are rich, universal, and quantitative mathematic properties shared by all of these chaotic systems.

We begin by introducing examples of chaos and their defining characteristics. 
The intuitive descriptions of chaos, as will be shown, can be rigorously described by Liapunov exponents and fractal dimensions.
After this, the iterative logistic map as a model of population is studied in details, which exhibit an interesting behavior of periodic bifurcations to chaos that is also observed among a large class of dynamical systems. 
The pinnacle of this report, presented in the last chapter, after developing some mathematically rigorous tools to study chaos, is the demonstrations of Feigenbaum's universal constants of bifurcation.
\end{abstract}
