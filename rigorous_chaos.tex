\section{Devaney's Definition}

\begin{defn}[Topologically Transitive]
	Let $J$ be a set with a topology.
	The function $f: J \rightarrow J$ is topologically transitive, if for all non-empty open sets $U_1, U_2 \in J$, there exist $k \in \bb{N}$ such that $f^k(U_1) \cap U_2 \neq \varnothing$. 
	Here $f^k$ denotes the composition $\underbrace{f \circ f \cdots \circ f}_{k \text{ times}}$
\end{defn}

\begin{defn}[Dense]
	Let $X$ be subset of the topological space $J$. 
	$X$ is dense if $\overline{X} = J$, where $\overline{X}$ denote the closure of $X$.
\end{defn}

\begin{defn}[Periodic Point]
	A point $x$ is a period of period $k$ if $f^k(x)=x$.
\end{defn}

\begin{defn}[Sensitive Dependence on Initial Condition]
	Let $J$ be a metric space with the metric $d: J \times J \rightarrow \bb{R}$.

	$f: J \rightarrow J$ has sensitive dependence on initial condition if there exists some $\delta > 0$ such that for all $x \in J$ and any neighbourhood $X$ containing $x$, there exists $y \in X$ and $n \in \bb{N}$ such that $d(f^n(x), f^n(y)) > \delta$.
\end{defn}

\begin{defn}[Chaos, Devaney's Definition]\label{def:Devaney_definition_for_chaos}
	Let $V$ be a metric space.
	A function $f: V \rightarrow V$ is chaotic if 
	\begin{enumerate}
		\item the set of periodic points are dense in $V$,
		\item $f$ is topologically transitive,
		\item $f$ has sensitive dependence on initial condition. 
	\end{enumerate}
	
\end{defn}


The definition here is attributed to Devaney \cite{Devaney_green_book_chaos_definition}. 
The first two requirements are purely topological, while the last one requires a metric. 
While in most circumstances topological properties are more generalised than the metric properties, here, surprisingly, as long as $V$ is a metric space (which is part of the assumption), the first two implies the third\cite{Banks}. 
As a result, we only need to check the first two conditions to determine if a function is chaotic.

\begin{thm}[Criterion For Chaotic Function]
	Let $V$ be a metric space. 
	Function $f: V \rightarrow V$ is chaotic as defined in \ref{def:Devaney_definition_for_chaos}
	if and only if its the periodic points are dense $V$ and it is topologically transitive,
\end{thm}

% TODO: Say more about doubling map. 
% TODO: graph, say more in introduction, etc?
% The doubling map is topologically transitive to the map in $S^1$

\begin{example}\label{ex_doubling_map}
	Recall the doubling map $f: [0,1) \rightarrow [0,1)$
	$$
	f(x) = 
		\begin{cases}
			2x &\text{ if } 2x < 1 \\
			2x -1 &\text{ if } 2x > 1
		\end{cases}
	$$
	% TODO: more explanation
	This map can also be regarded as doubling the angles on the unit cycle: $g: S^1 \rightarrow S^1$ given by $ g(\theta) = 2 \theta$.
	
	The doubling map is \textit{chaotic}.

	All non-zero rational points $\frac{p}{q} \in [0,1)$ with odd denominater are periodic points for $f$. 
	The set of these points are dense.
	To show the periodicity, note the images of $\frac{p}{q}$ under repetitive applications of $f$ are $\frac{p}{q}, \frac{2p}{q}, \cdots, \frac{2^k p}{q}, \cdots$.
	We can regard all the numerators as the equivalence classes module $q$.
	Since this sequence is infinite and there are only finitely many possibilities for the numerators, some of the numerators must coincide. Let them be $2^k p$ and $2^{k'} p$. 
	Without loss of generacity let $k > k'$, and
	$$
	2^k p \equiv 2^{k'} p \mod q \implies 
	2^{k-k'} p \equiv p \mod q \text{ (as $q$ is odd)},
	$$
	which measn $f^{k-k'}(\frac{p}{q}) = \frac{p}{q}$, i.e., $\frac{p}{q}$ is a point of period $k - k'$.

	To show $f$ is topologically transitive is even easier. 
	For any non-empty open set $U \in [0,1)$, by definition there exists an open interval $J = (x, x+ \delta) \subseteq U$. 
	$J$ has diameter $\delta$, $f(J)$ has diameter $2 \delta$, and $f^k(J)$, $2^k \delta$. 
	Since the length of $[0,1)$ is $1$, after some finite iteration $f^k(J)$ would covers all of $[0,1)$ and intersect any other open sets.

	By generalising this proof it is clear any maps in the form 
	$$
	g: S^1 \rightarrow S^1; g(\theta) = r \theta, r \in R
	$$
	are chaotic.
\end{example}

To prove the doubling map is chaotic is a gentle exercise. 
To directly find the points of periodicity and prove that a general function is topologically transitive is difficult.
Instead, we can circumvent these challenges by exploiting certain topological properties introduced in the next section.

\section{Topological Conjugacy}

\begin{defn}[Topological Conjugacy]
Funcitons $f: X \rightarrow X$, $g:  Y \rightarrow Y$ are topologically conjugate if there exist a homeomorphism $\phi: X \rightarrow Y$ such that 
$\phi \circ f = g \circ \phi$,
i.e, the following diagram commutes.
\begin{center}
    \begin{tikzcd}
        X \arrow[r, "f"] \arrow[d, "\phi" swap] & X \arrow[d, "\phi"] \\
        Y \arrow[r, "g"] & Y
    \end{tikzcd}
\end{center}

The maps $f,g$ are semiconjugate if there exists a continuous surjection, $\psi$ such that $\psi \circ f = g \circ \psi$.
\end{defn}

% TODO: Shall we introduce a notation for topological conjugacy? 
% no notation is provided by Devaney or Wikipedia

To rephrase a definition, $f,g$ are conjugate if there exists and homeomorphism $\phi$ such that $f = \phi^{-1} \circ g \circ \phi$.
So $f^{n} = \phi^{-1} \circ g^{n} \circ \phi$; that is $f^n$ is conjugate to $g^n$.

Topological conjugacy is an equivalence condition. 
$f$ is conjugate to itself by identity map. 
$\phi \circ f = g \circ \phi$ implies $f \circ \phi^{-1} = \phi^{-1} \circ g$, so $g$ is also conjugate to $f$.
Lastly, if $f$  is conjugate to $g$ via $\phi$, and $g$ is conjugate to $h$ via $\phi'$, $\phi' \circ \phi \circ f = \phi' g \circ \phi = h \circ \phi'' $, which means $f$ is conjugate to $h$ via $h' \circ h$.

Topological semiconjugacy only requires a continuous surjection, $\psi$, such that $\psi \circ f = g \circ \psi$. 
As $\psi$ may not have inverset semiconjugacy can not be an equivalence condition.
Nevertheless, if $f$ is semiconjugate to $g$, so is $f^{k}$ to $g^{k}$ for any $k \in \bb{N}$. 
This is because 
$$
	\psi \circ f\circ \cdots \circ f = g \circ \psi \circ f \cdots \circ f = \cdots = g^k \circ \psi
$$

 % TODO: verify this is true
If $f$ is conjugate to $g$, necessarily $f$ and semi-conjugate to $g$ and $g$ is semiconjugate to $f$. 
However, if $f$ is semiconjugate to $g$ and $g$ is semiconjugate to $f$, $f$ is not necessarily conjugate to $g$. 

Chaotic behaviors are \emph{preserved} by conjugacy.

\begin{thm}[Semiconjugacy preserves Chaos]\label{th_semicong_chaos}
	If $f$ is semiconjugate to $g$ and $f$ is chaotic, $g$ also is.
\end{thm}
% TODO: can g chaotic implies f also is if f is semiconjugate to g?
% FIND counter example

\begin{proof}
	Let $f: X \rightarrow X$, $g:  Y \rightarrow Y$, and let $\psi$ be the promised continuous surjection such that
	$\psi \circ f = g \circ \psi$. 
	Since $\psi$ is conituously surjective, for any non-empty open set $M \in Y$, $\psi^{-1}(M)$ is an non-empty open set in $X$.

	Let us first prove that the set of periodic points of $g$ is ddense.
	For any $x$ such that $f^k(x) = x$, notice that $g^k \circ \psi (x) = \psi \circ f^k (x) = \psi(x)$. 
	This means, if the set of periodic points of $f$ is $X$, $\psi(X)$ is a sub set of the set of periodic point of $g$.

	For the sake of contradiction assuming $\psi(X)$ is not dense in $Y$. 
	This means there is a open set $M \subset Y$ such that $M \cup \psi(X) = \emptyset$.
	The preimage $\psi^{-1}(M)$ is an non-empty open set of $X$, and by assumption, it does not contains any periodic points of $f$, which is a contradiction.

	To prove that $g$ is topologically transitive, consider any non-empty open set $M, N \in Y$, and $\psi^{-1}(M), \psi^{-1}(N)$ are non-empty open set in $X$. 
	By assumption there exists some $k$ such that $f^k(\psi^{-1}(M)) \cap f^k(\psi^{-1}(N)) \neq \emptyset$, this means 
	\begin{align*}
		g^k(M) \cap g^k(N) = \psi \circ f^k(\psi^{-1}(M)) \cap \psi \circ f^k(\psi^{-1}(N)) \\
		\subset \psi( f^k(\psi^{-1}(M)) \cap  f^k(\psi^{-1}(N))) \neq \emptyset
	\end{align*}
\end{proof}

Since conjugacy impies semi-conjugacy on both directions, we have the following important proposition.

\begin{prop}[Conjugacy Class share chaotic behavior]\label{prop_conj_chaos}
	If $f$ is conjugate to $g$, $f$ is chaotic if and only if $g$ is chaotic.
\end{prop}

There are abundant examples of topological conjugacy.

\begin{example}
	If $f$ is conjugate to $g$ via $\phi$, and $g$ is conjugate to $h$ via $\psi$, $f$ is conjugate to $h$ via $\psi \circ \phi$.
	\begin{center}
	    \begin{tikzcd}
	        X \arrow[r, "f"] \arrow[d, "\phi" swap] & X \arrow[d, "\phi"] \\
	        Y \arrow[r, "g"] \arrow[d, "\psi" swap] & Y \arrow[d, "\psi"] \\
	        Z \arrow[r, "h"] & Z \\
	    \end{tikzcd}
	\end{center}
\end{example}

\begin{example}
	For open interval $[-\epsilon_1, \epsilon_2] \in \bb{R}$. 
	$f: [-\epsilon_1, \epsilon_2] \rightarrow [-\epsilon_1, \epsilon_2]$
	is conjugate to 
	$g: [-\epsilon_1 + a, \epsilon_2 + a] \rightarrow  [-\epsilon_1 + a, \epsilon_2 + a]$
	where $g(x)= f(x -a) + a$ via $\phi(x) = x + a$.

	For a concrete example, consider the map 
	$f: [0,1] \rightarrow [0,1]$
	defined as $f(x) = 4x(1-x)$,
	which is conjugate to 
	$g(x): [-\frac{1}{2}, \frac{1}{2}] \rightarrow [-\frac{1}{2}, \frac{1}{2}]$
	defined as $g(x) = -4x^2 + \frac{1}{2}$ 
	through $\phi(x) = x -\frac{1}{2}$, i.e., $ \phi \circ f=  g \circ \phi$.
\end{example}

\begin{example}
	For closed interval $[-\epsilon_1, \epsilon_2] \in \bb{R}$. 
	$f: [-\epsilon_1, \epsilon_2] \rightarrow [-\epsilon_1, \epsilon_2]$
	is conjugate to 
	$g: [-\frac{\epsilon_1}{a}, \frac{\epsilon_2}{a}] \rightarrow [-\frac{\epsilon_1}{a}, \frac{\epsilon_2}{a}]$
	where $g(x)= \frac{1}{a}(ax)$ via $\phi(x) = ax$.


	As another concrete example, the map $f: [0,1] \rightarrow [0,1]$
	defined as $f(x) = -4x^2 + \frac{1}{2})$,
	is conjugate to 
	$g(x): [-1, 1] \rightarrow [-1, 1]$
	defined as $g(x) = 2x^2 - 1$
	through $\phi(x) = -2x$.
\end{example}

\begin{example}
	Consider the doubling map $f: [0,\pi) \rightarrow [0, \pi)$ defined thus: 
	$$
	f(\theta) = 
		\begin{cases}
			2 \theta &\text{ if } 2x <  \pi \\
			2 \theta - \pi &\text{ if } 2x > \pi
		\end{cases}
	$$
	
	This map is doubling the angle of the unit circle.

	$f$ is semi-conjugate to $g: [-1, 1] \rightarrow [-1, 1]$ defined as $g(x) = 2x^2 - 1$ through $\cos(x)$.

	Example \ref{ex_doubling_map} shows that the doubling map is chaotic, so is $g$.
\end{example}

\begin{example}\label{ex_logistic_at_4}\label{ex_logistic_and_doubling}
	The above 3 examples shows that the doubling map is semi-conjugate to the logistic map $L_1(x) = 4x(1-x)$.
\end{example}

\begin{example}
	The logistic map $L_1(x) = 4x(1-x)$ in the interval $[0,1]$ is topologically conjugate to the tent map defined as 
	\begin{equation}
		f(x) = 
		\begin{cases}
			2x   &\text{ if } 0<x \leq \frac{1}{2} \\ 
			2-2x &\text{ if } \frac{1}{2} < x \leq 1
		\end{cases}
	\end{equation}
	via the map $h: [0,1] \rightarrow [0,1]$ defined as $h(x) = \sin^2(\frac{\pi x}{2})$.
	This can be verified by the simple computation $L_1 \circ h = h \circ f$.
\end{example}
